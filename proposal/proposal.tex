\documentclass{article}

\usepackage[inline]{enumitem}

\usepackage{url}
\usepackage{graphicx}
\graphicspath{ {./figures/} }

\title{Decentralised Mining Pool for Bitcoin (Draft 0.1)}
\author{Kulpreet Singh}
\date{February 2021}

\begin{document}

\maketitle

\begin{abstract}
  Bitcoin p2pool's usage has steadily declined over the years,
  negatively impacting bitcoin's ability to remain decentralised. The
  primary problems with p2pool were twofold. First, the variance in
  earnings for miners didn't reduce with the increase in hashrate
  particapting in p2pool. Secondly, making payouts to miners required
  a linearly increasing blockspace with the increase in the numer of
  miners on p2pool. Building a DAG of miner's shares and the use of
  payment channels are two proposals trying to alleviate these
  problems faced by p2pool. In this document, we present a unified
  solution that uses a directed acyclic graph to track miners shares
  and uses payments channels to reward miners. The shares calculation
  can be carried out by any node on the p2p, and the rewards are paid
  out by a pool operator. Using the payment channels construction
  neither the pool operator nor the miners can cheat each other. We
  show that our approach is incentives compatible and reduces variance
  in earnings for miners. We also propose a solution for trading the
  miner's proof of work for BTC in an open market.
\end{abstract}
   
\section{Motivation}

P2Pool~\cite{p2pool:wiki} helped decentralise bitcoin by enabling
miners to select which transactions they mined and thus avoid any
potential censorship by pool operators. However, the construction used
by P2Pool faced a number of problems that lead to miners abandonding
the pool.

\begin{enumerate}
  \item Large variance in earnings for miners.
  \item Large number of dead on arrival blocks.
  \item Large block space requirement.
\end{enumerate}

The first two problems are a direct consequence of the shares block
rate limited to 30 seconds. With only one block possible every thirty
seconds, any increase of hashrate on P2Pool resulted in shares
competing to be the next block in the p2pool chain.

There is a clear tension in play here, increasing the block rate
frequency doesn't scale the throughput of the pool in terms of number
of shares found, as most of them are orphaned and the miners not being
rewarded for those orphans. Ethereum's inclusive
protocols~\cite{inclusive-protocols} help alleviate the problem for
the Ethereum blockchain, where small pools can work with a reduced
variance in their rewards as shown in the analysis by
McElrath~\cite{mcelrath:variance}.

Knowing the challenges faced by P2Pool, we list the goals of a new
decentralised mining pool as:

\begin{enumerate}
  \item Lower variance for miners to enable the long tail of independent miners.
  \item Indepedent miners that build their own blocks.
  \item Payouts for miners with constant size block space requirement.
  \item Provide building blocks for a hash rate futures market.
\end{enumerate}

\section{Current Proposals}

TerraHash Coin~\cite{mcelrath:variance}, Jute~\cite{jute}
and~\cite{spectre} are some of the attempts to use a DAG for faster
block times. However, these works focus on changing the consensus
layer of bitcoin itself. The ideas in these proposals allowed miners
to produce shares that have conflicting transactions and then apply
rules to find a set of transactions acceptable at various cuts of the
DAG.\@

Instead we propose to build a DAG of miner shares to enable faster
block times and using this DAG for calculating distribution of payouts
between miners. We then propose using payment channels as defined by
Belcher~\cite{channels-for-rewards} to avoid using block space for
making payouts to miners.

Apart from the work for increasing block rates using DAGs,
Belcher~\cite{channels-for-rewards} proposes a different idea to help
decentralise mining deals with the problem of a paying miners from a
decentralised mining pool. P2Pool uses the coinbase tranasaction of a
block to pay out miners. Belcher shows how a scheme can be constructed
using payment channels between federated hubs to pay miners after a
block has been successfully mined. The payouts can be paid after a
long enough period, similar to the 100 blocks requirements for
spending from coinbase transactions. Miners can register with hubs
where bitcoin has been locked in to open payment channels to
miners. The construction presented by Belcher shows how both miners
and hubs can't cheat and how the funders of the hub can earn a reward
for funding the payment channels.

The ideas of the Hash Rate Futures and Payment Channels for Rewards
Payouts together present a potential path for rebooting P2Pool. In the
rest of the document we present a slightly modified version of
TerraHash Coin and show how the various components can work toegether.

\section{Decentralised Bitcoin Mining}

In this section we present a modified version of TerraHash Coin and
show how it can use payments channels with hubs to deliver a
decentralised mining pool for bitcoin.

\subsection{A DAG of Shares}

The braiding the blockchain proposal~\cite{mcelrath:variance} shows
how smaller more frequent blocks can form a directed acyclic graph
(DAG) of blocks, with each block pointing to one or more one previous
blocks. Blocks in TerraHash Coin can have transactions repeated in
different blocks. The proposal describes how repeated and potentially
some double spend transactions can be resolved to decide on the state
of the ledger at any cut of the DAG.\@

The rewards that miners earn in the proposal is a coin native to the
braid blockchain and is called TerraHash Coin. This coin can then be
swapped for Bitcoin. The proposal doesn't yet define how this native
coin will be swapped by bitcoin. Some of the suggestions under
discussion include using atomic swaps, burning the TerrahHash Coin, or
using financial instruments like futures of the bitcoin's hash
rate to swap TerraHash Coins for BTC.\@

We propose taking a slightly different approach, where the blocks of a
DAG represent shares of the mining pool, use payment channels between
miners and a hub for distributing payouts in proportion to the work done
by the miners.

Each miner builds their own block, selecting transactions as they
want, we call this block the \textsc{work}. The description of
\textsc{work} is then disseminated to the p2p network of miners using
the compact block specifications~\cite{compact-blocks}.

The miner then starts mining on \textsc{work} and generates
\textsc{share}. Each \textsc{share} is mined at a difficulty level
chosen by the miner. This difficutly can be dynamically chosen by the
miner after each \textsc{share}, depending on what the miner observes
on the p2p network. This dynamic adjustment allows miners to adjust
the rate at which they produce \textsc{share}s. [TODO: Build a model to
recommend an emission rate of these shares.]

Figure~\ref{fig:work-share} shows the relationship between
\textsc{work} and its \textsc{share}s. Each \textsc{work} created by a
miner has multiple \textsc{share}s and they are both broadcast on the
p2p network.

\begin{figure}[h]
  \begin{center}
    \includegraphics[width=1.0\textwidth]{work-share}
    \caption{Each \textsc{work} generated and shared by a miner is then
      followed by the \textsc{share}s the miner finds.}\label{fig:work-share}
    \end{center}
\end{figure}

The nodes in the DAG are \textsc{share}s mined at varied difficulty
levels. Each \textsc{share} that matches or exceeds the current
bitcoin difficulty starts a new $epoch$ for the p2p mining
pool. Figure~\ref{fig:epoch} shows $l$ and $r$ as the two valid
bitcoin blocks that have been mined such that they meet bitcoin's
difficulty at the time the block was mined, and all the blocks between
$l$ and $r$ are in the same $epoch$.

\begin{figure}[h]
  \includegraphics[width=1.0\textwidth]{epoch}
  \caption{A epoch is defined as all the \textsc{share}s mined between two
    bitcoin blocks. Here all the \textsc{share}s between $l$ and $r$ are in
    the same $epoch$.}\label{fig:epoch}
\end{figure}

When a miner starts working on a \textsc{share} it includes a
reference to the highest known \textsc{share} from all other miners
that the miner has receieved valid shares from. Note, the miner also
has access to \textsc{work} blocks from all participating miners. If a
miner doesn't have the \textsc{work} block from another miner, then it
rejects any \textsc{share}s received from the other miner. This
requires that miners is to the deteriment of the other miners as we
show in Section~\ref{sec:rewards}.

In the next section we then describe how all peers compute their fair
share of profits using the DAG of shares. We show how our reward
computation algorithm is incentives
compatible~\cite{incentives-compatible}.

%% - block datastructure
%%   - coinbase reward goes to Hub's address
%%   - ignore this for now, we'll show how the reward is distributed in a
%%     trustless manner to all miners.
%% - DAG of shares
%%   - Epochs: start from and end with bitcoin block
%%   - Epoch ends when a valid bitcoin block with current bitcoin
%%     difficulty is found
%%   - Immediately send mined bitcoin block to bitcoin network
%%   - Hub will distribute the reward in around 100 blocks time
%% - Use compact blocks to inform others about the block we are mining
%%   - Other miners can decide to include your block as a previous block
%%     or not, whenever we find a solution and announce it.
%%   - We need to model the network traffic and latency
%% - Keep mining the same block, until end of epoch
%%   - For each block mined, share the solution with p2p network
%%   - Only if the mined block meets the current bitcoin difficulty
  
  
\subsection{Incentives Compatible Rewards}\label{sec:rewards}

Each participating node, which includes the miners and the hub, is
able to the see the DAG of \textsc{share}s as broadcast on the
network. Each \textsc{share} includes a reference to the blocks the
miner was aware of when the \textsc{share} was found. The reason for
doing so is simple. If a miner $a$ doesn't include the \textsc{share}s
of miner $b$, then $b$ has a clear signal to stop including the
\textsc{share}s of $a$, and as we will see a miner wants that their
\textsc{share}s are referenced by other miners as only then they will
be rewarded for their work.

The incentive in lay terms is that all miners should honestly include
the \textsc{share}s discovered by other miners, as otherwise they will
most likely be excluded by other miners and they will lose the
opportunity to be rewarded for their work. We call this the
degenerative case of ``isolated miners'' and argue that miners have no
incentives to act in this manner. Figure~\ref{fig:isolated-miners}
shows a DAG where all three miners $a$, $b$ and $c$ are working
independently. In such a situation when the miner $a$ discovers a
share and the reward is not shared with any other miner.

\begin{figure}[h]
  \begin{center}
    \includegraphics[width=0.5\textwidth]{isolated-miners}
    \caption{$a$ discovers a share and the reward is not shared with any
      other miner.}\label{fig:isolated-miners}
  \end{center}    
\end{figure}

With the above understanding of why miners will co-operate, we now
state the rules to calculate how the block reward should be divided
between miners.

\begin{enumerate}
  \item Traverse the DAG in reverse order from the \textsc{share} that
    found the latest bitcoin block to the previous bitcoin block found
    and collect a set of shares.
  \item From the above set of shares remove all shares that don't have
    a reverse path to the previous bitcoin block.
  \item Distribute the reward between miners weighted by the sum of
    the diffcultly of all \textsc{share}s found by miners.
\end{enumerate}

As an example consider the p2p network of miners $a$, $b$ and $c$ with
the DAG of shares as shown in Figure~\ref{fig:shares-dag}. In the DAG
the set of shares that receive reward proporitional to their
difficulty are $\{a_i..a_5, b_1..b_3\}$. The shares $\{c_1..c_3\}$ do
not receive any reward as they are not reachable from the bitcoin
block, $a_5$, even if they are reachable from $l$.

For the second bitcoin block $b_5$ only the miners $a$ and $b$ receive
rewards in proportion to the difficulties of their shares $\{b_4, b_5,
a_6\}$. $c$ doesn't receieve any reward for $c4$ as it is doesn't
include a reference to the last found bitcoin block $a_5$.

\begin{figure}[h]
  \begin{center}
    \includegraphics[width=1.0\textwidth]{shares-dag}
    \caption{Two epochs in a DAG of shares mined by three mines ---
      $a$, $b$ and $c$. The shares in grey meet the bitcoin difficulty
      at the time they were mined.}\label{fig:shares-dag}
  \end{center}
\end{figure}

With the above rules, it should be easy to prove that the rule reward
is an incentives compatible reward function as defined by
[ref:incentives compatibility], and we present an outline of proofs
that will be be formalised in future work.

\begin{description}
  \item [Incentive Compatibility] Given the rules above, if a miner
    finds a bitcoin block the miner wants to get maximum reward
    possible based on all the shares it has found and therefore is
    incentivized to announce their \textsc{share} as soon as they find
    it.
  \item [Proportional Payments] Since rewards are calculated at the
    end of an epoch, that is when all the valid shares found by all
    miners have been discovered, all miners are guaranteed payments
    for the shares that have reached the miner who found the block.
  \item [Budget Balanced] Again, since rewards are paid at the end of
    the epoch, a hub pays out rewards without losing or retaining any
    amount.
\end{description}

%% - Rewards computation
%%   - Rewards distribution
%%   - Incentives Compatible

\subsection{Payment Channels}

The Hubs and payments channels proposal is based on the proposal
described by Belcher~\cite{channels-for-rewards} with a change that
there is a single hub and it is protected from DDoS by using
techniques to enable responder anonymity in p2p
networks~\cite{responder-anonymity:file-sharing}. With this change the
hub acts just like any other miner on the network to be completely
undifferentiated from other miners, thus protecting itself from DDoS,
by requiring an attacker to attack the whole network.

The only difference
is that Hubs calculate and distribute the rewards according to the
incentives compatible rewards scheme described above. We also adopt
the proposal that rewards are paid out after a 100 blocks of a block
being mined.

We would like to extend the proposal to include funding of hubs by
more than one party in a trustless manner, so that parties with small
BTC parties can engage in the peer to peer mining economy. This will
encourage faster adoption of our p2p mining network and make the
network stronger against DDoS attacks with multiple hub operators
available online.


%% \subsection{Implementation Plan}

%% The key components to build include:

%% 1. P2P shares network.
%%    1. Peers exchange both WORK blocks and SHARE
%%    blocks. The network is permissionless, i.e. anyone can join and
%%    start sharing their WORK and SHARE blocks.
%% 2. Rewards Computation.
%%    1. An algorithm to compute rewards.
%% 3. Hub.
%%    1. Create channels with p2p miners.
%%    2. Compute reward and distribute them.
%%    3. Connect to other hubs using a p2p or a rqelay network.
   

\section{Future Work}

\subsection{Proofs}

We want to use the model presented by Boneh et.al.\ to provide proofs
for how the rewards distribution is incentives compatible.

\subsection{Simulations}
Before we work on implementing they system, our next step is to
simulate p2p mining network using ns-3 [ref] and make informed
decisions about how large a network each hub will want to support. The
observations we want to make are how large a p2p network can be
sustained without an increase in work lost by miners. Each hub and p2p
network can grow as long as miners are communicate WORK and SHARES
with each other with bounded latency and can limit their lost
work. With a simulation we want to find out the bounds of these.

\subsection{Specifications and Implementation}

We want to specify the p2p protocol messages and the rules more
precisely. We also plan to implement the specifications and we expect
the two tasks to proceed hand in hand.

\bibliography{proposal} 
\bibliographystyle{acm}

\end{document}
