\documentclass{article}

\usepackage[inline]{enumitem}

\usepackage{url}
\usepackage{graphicx}
\graphicspath{ {./figures/} }
\usepackage{amsfonts}% to get the \mathbb alphabet
\usepackage{booktabs}
\usepackage{multirow}


\title{Decentralised Mining Pool for Bitcoin (Draft 0.1)}
\author{Kulpreet Singh}
\date{February 2021}

\begin{document}

\maketitle

\begin{abstract}
  Bitcoin p2pool's usage has steadily declined over the years,
  negatively impacting bitcoin's ability to remain decentralised. The
  primary problems with p2pool were twofold. First, the variance in
  earnings for miners didn't reduce with the increase in hashrate
  participating in p2pool. Secondly, making payouts to miners required
  a linearly increasing blockspace with the increase in the number of
  miners on p2pool. Building a DAG of miner's shares and the use of
  payment channels are two proposals trying to alleviate these
  problems faced by p2pool. In this document, we present a unified
  solution that uses a directed acyclic graph to track miners shares
  and uses payments channels to reward miners. The shares calculation
  can be carried out by any node on the p2p, and the rewards are paid
  out by a pool operator. Using the payment channels construction
  neither the pool operator nor the miners can cheat each other. We
  show that our approach is incentives compatible and reduces variance
  in earnings for miners. We also propose a solution for trading the
  miner's proof of work for BTC in an open market.
\end{abstract}
   
\section{Motivation}

P2Pool~\cite{p2pool:wiki} helped decentralise bitcoin by enabling
miners to select which transactions they mined and thus avoid any
potential censorship by pool operators. However, the construction used
by P2Pool faced a number of problems that lead to miners abandoning
the pool.

\begin{enumerate}
  \item Large variance in earnings for miners.
  \item Large number of dead on arrival blocks.
  \item Large block space requirement.
\end{enumerate}

The first two problems are a direct consequence of the shares block
rate limited to 30 seconds. With only one block possible every thirty
seconds, any increase of hashrate on P2Pool resulted in shares
competing to be the next block in the p2pool chain.

There is a clear tension in play here, increasing the block rate
frequency doesn't scale the throughput of the pool in terms of number
of shares found, as most of them are orphaned and the miners not being
rewarded for those orphans. Ethereum's inclusive
protocols~\cite{inclusive-protocols} help alleviate the problem for
the Ethereum blockchain, where small pools can work with a reduced
variance in their rewards as shown in the analysis by
McElrath~\cite{mcelrath:variance}.

Knowing the challenges faced by P2Pool, we list the goals of a new
decentralised mining pool as:

\begin{enumerate}
  \item Lower variance for miners to enable the long tail of independent miners.
  \item Independent miners that build their own blocks.
  \item Payouts for miners with constant size block space requirement.
  \item Provide building blocks for a hash rate futures market.
\end{enumerate}

\section{Current Proposals}

TerraHash Coin~\cite{mcelrath:variance}, Jute~\cite{jute}
and~\cite{spectre} are some of the attempts to use a DAG for faster
block times. However, these works focus on changing the consensus
layer of bitcoin itself. The ideas in these proposals allowed miners
to produce shares that have conflicting transactions and then apply
rules to find a set of transactions acceptable at various cuts of the
DAG.\@

Instead we propose to build a DAG of miner shares to enable faster
block times and using this DAG for calculating distribution of payouts
between miners. We then propose using payment channels as defined by
Belcher~\cite{channels-for-rewards} to avoid using block space for
making payouts to miners.

Apart from the work for increasing block rates using DAGs,
Belcher~\cite{channels-for-rewards} proposes a different idea to help
decentralise mining deals with the problem of a paying miners from a
decentralised mining pool. P2Pool uses the coinbase transaction of a
block to pay out miners. Belcher shows how a scheme can be constructed
using payment channels between federated hubs to pay miners after a
block has been successfully mined. The payouts can be paid after a
long enough period, similar to the 100 blocks requirements for
spending from coinbase transactions. Miners can register with hubs
where bitcoin has been locked in to open payment channels to
miners. The construction presented by Belcher shows how both miners
and hubs can't cheat and how the funders of the hub can earn a reward
for funding the payment channels.

The ideas of the Hash Rate Futures and Payment Channels for Rewards
Payouts together present a potential path for rebooting P2Pool. In the
rest of the document we present a slightly modified version of
TerraHash Coin and show how the various components can work together.

\section{Decentralised Bitcoin Mining}

In this section we present a modified version of TerraHash Coin and
show how it can use payments channels with hubs to deliver a
decentralised mining pool for bitcoin.

\subsection{A DAG of Shares}

The braiding the blockchain proposal~\cite{mcelrath:variance} shows
how smaller more frequent blocks can form a directed acyclic graph
(DAG) of blocks, with each block pointing to one or more one previous
blocks. Blocks in TerraHash Coin can have transactions repeated in
different blocks. The proposal describes how repeated and potentially
some double spend transactions can be resolved to decide on the state
of the ledger at any cut of the DAG.\@

The rewards that miners earn in the proposal is a coin native to the
braid blockchain and is called TerraHash Coin. This coin can then be
swapped for Bitcoin. The proposal doesn't yet define how this native
coin will be swapped by bitcoin. Some of the suggestions under
discussion include using atomic swaps, burning the TerraHash Coin, or
using financial instruments like futures of the bitcoin's hash
rate to swap TerraHash Coins for BTC.\@

We propose taking a slightly different approach, where the blocks of a
DAG represent shares of the mining pool, use payment channels between
miners and a hub for distributing payouts in proportion to the work done
by the miners.

Each miner builds their own block, selecting transactions as they
want, we call this block the \textsc{work}. The description of
\textsc{work} is then disseminated to the p2p network of miners using
the compact block specifications~\cite{compact-blocks}.

The miner then starts mining on \textsc{work} and generates
\textsc{share}. Each \textsc{share} is mined at a difficulty level
chosen by the miner. This difficulty can be dynamically chosen by the
miner after each \textsc{share}, depending on what the miner observes
on the p2p network. This dynamic adjustment allows miners to adjust
the rate at which they produce \textsc{share}s. [TODO: Build a model to
recommend an emission rate of these shares.]

Figure~\ref{fig:work-share} shows the relationship between
\textsc{work} and its \textsc{share}s. Each \textsc{work} created by a
miner has multiple \textsc{share}s and they are both broadcast on the
p2p network.

\begin{figure}[h]
  \begin{center}
    \includegraphics[width=1.0\textwidth]{work-share}
    \caption{Each \textsc{work} generated and shared by a miner is then
      followed by the \textsc{share}s the miner finds.}\label{fig:work-share}
    \end{center}
\end{figure}

The nodes in the DAG are \textsc{share}s mined at varied difficulty
levels. Each \textsc{share} that matches or exceeds the current
bitcoin difficulty starts a new $epoch$ for the p2p mining
pool. Figure~\ref{fig:epoch} shows $l$ and $r$ as the two valid
bitcoin blocks that have been mined such that they meet bitcoin's
difficulty at the time the block was mined, and all the blocks between
$l$ and $r$ are in the same $epoch$.

\begin{figure}[h]
  \includegraphics[width=1.0\textwidth]{epoch}
  \caption{A epoch is defined as all the \textsc{share}s mined between two
    bitcoin blocks. Here all the \textsc{share}s between $l$ and $r$ are in
    the same $epoch$.}\label{fig:epoch}
\end{figure}

When a miner starts working on a \textsc{share} it includes a
reference to the highest known \textsc{share} from all other miners
that the miner has received valid shares from. Note, the miner also
has access to \textsc{work} blocks from all participating miners. If a
miner doesn't have the \textsc{work} block from another miner, then it
rejects any \textsc{share}s received from the other miner. This
requires that miners is to the detriment of the other miners as we
show in Section~\ref{sec:rewards}.

In the next section we then describe how all peers compute their fair
share of profits using the DAG of shares. We show how our reward
computation algorithm is incentives
compatible~\cite{incentives-compatible}.

%% - block datastructure
%%   - coinbase reward goes to Hub's address
%%   - ignore this for now, we'll show how the reward is distributed in a
%%     trustless manner to all miners.
%% - DAG of shares
%%   - Epochs: start from and end with bitcoin block
%%   - Epoch ends when a valid bitcoin block with current bitcoin
%%     difficulty is found
%%   - Immediately send mined bitcoin block to bitcoin network
%%   - Hub will distribute the reward in around 100 blocks time
%% - Use compact blocks to inform others about the block we are mining
%%   - Other miners can decide to include your block as a previous block
%%     or not, whenever we find a solution and announce it.
%%   - We need to model the network traffic and latency
%% - Keep mining the same block, until end of epoch
%%   - For each block mined, share the solution with p2p network
%%   - Only if the mined block meets the current bitcoin difficulty
  
  
\subsection{Incentives Compatible Rewards}\label{sec:rewards}

Each participating node, which includes the miners and the hub, is
able to the see the DAG of \textsc{share}s as broadcast on the
network. Each \textsc{share} includes a reference to the blocks the
miner was aware of when the \textsc{share} was found. The reason for
doing so is simple. If a miner $a$ doesn't include the \textsc{share}s
of miner $b$, then $b$ has a clear signal to stop including the
\textsc{share}s of $a$, and as we will see a miner wants that their
\textsc{share}s are referenced by other miners as only then they will
be rewarded for their work.

The incentive in lay terms is that all miners should honestly include
the \textsc{share}s discovered by other miners, as otherwise they will
most likely be excluded by other miners and they will lose the
opportunity to be rewarded for their work. We call this the
degenerative case of ``isolated miners'' and argue that miners have no
incentives to act in this manner. Figure~\ref{fig:isolated-miners}
shows a DAG where all three miners $a$, $b$ and $c$ are working
independently. In such a situation when the miner $a$ discovers a
share and the reward is not shared with any other miner.

\begin{figure}[h]
  \begin{center}
    \includegraphics[width=0.5\textwidth]{isolated-miners}
    \caption{$a$ discovers a share and the reward is not shared with any
      other miner.}\label{fig:isolated-miners}
  \end{center}    
\end{figure}

With the above understanding of why miners will co-operate, we now
state the rules to calculate how the block reward should be divided
between miners.

\begin{enumerate}
  \item Traverse the DAG in reverse order from the \textsc{share} that
    found the latest bitcoin block to the previous bitcoin block found
    and collect a set of shares.
  \item From the above set of shares remove all shares that don't have
    a reverse path to the previous bitcoin block.
  \item Distribute the reward between miners weighted by the sum of
    the difficultly of all \textsc{share}s found by miners.
\end{enumerate}

As an example consider the p2p network of miners $a$, $b$ and $c$ with
the DAG of shares as shown in Figure~\ref{fig:shares-dag}. In the DAG
the set of shares that receive reward proportional to their
difficulty are $\{a_i..a_5, b_1..b_3\}$. The shares $\{c_1..c_3\}$ do
not receive any reward as they are not reachable from the bitcoin
block, $a_5$, even if they are reachable from $l$.

For the second bitcoin block $b_5$ only the miners $a$ and $b$ receive
rewards in proportion to the difficulties of their shares $\{b_4, b_5,
a_6\}$. $c$ doesn't receive any reward for $c4$ as it is doesn't
include a reference to the last found bitcoin block $a_5$.

\begin{figure}[h]
  \begin{center}
    \includegraphics[width=1.0\textwidth]{shares-dag}
    \caption{Two epochs in a DAG of shares mined by three mines ---
      $a$, $b$ and $c$. The shares in grey meet the bitcoin difficulty
      at the time they were mined.}\label{fig:shares-dag}
  \end{center}
\end{figure}

With the above rules, it should be easy to prove that the rule reward
is an incentives compatible reward function as defined by
[ref:incentives compatibility], and we present an outline of proofs
that will be be formalised in future work.

\begin{description}
  \item [Incentive Compatibility] Given the rules above, if a miner
    finds a bitcoin block the miner wants to get maximum reward
    possible based on all the shares it has found and therefore is
    incentivized to announce their \textsc{share} as soon as they find
    it.
  \item [Proportional Payments] Since rewards are calculated at the
    end of an epoch, that is when all the valid shares found by all
    miners have been discovered, all miners are guaranteed payments
    for the shares that have reached the miner who found the block.
  \item [Budget Balanced] Again, since rewards are paid at the end of
    the epoch, a hub pays out rewards without losing or retaining any
    amount.
\end{description}

%% - Rewards computation
%%   - Rewards distribution
%%   - Incentives Compatible

\subsection{Payment Channels}

We propose using Payment Channels for paying miners based on the work
by Belcher~\cite{channels-for-rewards}. The construction of the
payments channels we present is similar to Belcher's construction, but
there are a few changes and we highlight before describing the
details.

\begin{itemize}
\item We use the DAG based scheme to distribute rewards among miners,
  so we avoid the problem of estimating block rewards before miners
  have created their shares.
\item We use a single hub, and prevent DDoS attacks using responder
  anonymity~\cite{responder-anonymity:file-sharing, liu2010rumor,
  responder-anonymity:p2p, gap-gnunet} techniques
  developed for P2P networks.
\end{itemize}

% TODO
% Question: How do we know the Hub is not creating coinbases with
% different preimages with each miner?

% Answer: Miners receive the pubkey for all other miners with the
% \textsc{work} announcement. Using this and the structure of the
% coinbase transactions, miners validate that the coinbase matches the
% hub's preimage, and the miner's pubkey.

\subsubsection{Coinbase}

We use Belcher~\cite{channels-for-rewards} construction where each
miner builds a coinbase transaction that can be spent in one of the
following three ways:

\begin{enumerate}
\item Co-operatively by Hub and Miner, or
\item The Hub with a hash lock for pre-image \verb|X|, or
\item The Miner that found the bitcoin block, but after waiting
  for six months.
\end{enumerate}

The scriptPubKey for the above conditions in the coinbase can be
written as:

\begin{table}
  \centering
  \begin{tabular}{ l }
    \bfseries Coinbase \\
    \midrule
    \verb|2 H M 2 CHECKMULTISIG| \\
    \verb|OR| \\
    \verb|hash(X) + Hub P2WPKH| \\
    \verb|OR| \\
    \verb|M and CHECKSEQUENCEVERIFY 6 months| \\ 
    \midrule
  \end{tabular}
  \caption{Coinbase transaction with hub and miner public keys.}\label{table:coinbase}
\end{table}

These conditions mean that the hub can not spend the coinbase without
revealing the pre-image $X$. This pre-image is included in the
construction of payment channels, as we will see in the next
section. This use of pre-image in both the coinbase and the payment
channel definition guarantees that miners get paid for their
accumulated payouts if the Hub defects.

\subsubsection{One-way Channels}

One-Way payment channels between hub and all miners allow miners to
receive payouts without consuming any blockspace in the bitcoin block
they mined. The use of payment channels is what makes braidpool scale
up without losing blockspace. Braidpool thus avoids losing the fees
that can be earned from the blockspace.

Just like in the early versions of proposal by Belcher we use one-way
payment channels. One-Way payment channels solve the problem of
aggregating a miner's payouts requiring a single blockchain
transaction for spending multiple payouts earned from their PoW
shares.

We use one-way instead of 2-way payment channels as an initial
implementation for Braidpool. If required we can switch to 2-way
channels~\cite{poon2016bitcoin} allowing miners to spend their mining
payouts over the lightning network. However, for now, we deliberately
stay away from the complexity of making the hub a lightning node. If
there is interest from miners to use the lightning network, we can
build that out in future.

Each miner has a one-way payment channel with the hub using a two of
two multisig with a time lock of six months. For each payout a miner
receives over the payment channel, the hub will charge an agreed upon
fees between the miner and the hub. Belcher's proposal recommends a
$0.1\%$ fees for the hub.

\subsubsection{Transactions}

The hub creates a funding channel locking an amount $R$ of
bitcoin. The miner then creates a refund transaction, spending the
funding transaction and sending the $R$ bitcoin back to the Hub. The
refund transaction has a locktime of six months allowing the miner to
accumulate their payout over the six month period. The protocol can be
extended to allow each miner to agree upon a locktime with the
hub. The trade-off will be between the fees charged by the hub and the
length of the locktime.

The funding transaction includes an input from the hub and an output
that can be spent in one of the two conditions shown in
Table~\ref{fund-tx}. \verb|H| and \verb|M| are the public keys for Hub
and Miner, they are called the co-operative keys by Belcher. While
\verb|H'| and \verb|M'| are alternative public keys for Hub and Miner,
and are called the uncooperative keys in the Belcher proposal.


\begin{table}
  \centering
  \begin{tabular}{ ll }
    \multicolumn{2}{c}{\bfseries Fund Transaction} \\
    \midrule
    \bfseries Input & \bfseries Output \\
    \midrule
    Hub's UTXO & \verb|2 H  M  2 CHECKMULTISIG| \\
    (Signed by the hub) & \verb|OR| \\
                    & \verb|2 H' M' 2 CHECKMULTISIG + Hash(X)| \\
    & (R coins) \\
    \midrule
  \end{tabular}
  \caption{Fund transaction for payment channel between hub and miner.}\label{fund-tx}
\end{table}


The hub doesn't broadcast the funding transaction instead it waits for
the miner to create refund transaction. This is the same as any other
timelocked one-way channel construction, i.e.\ the Miner creates a
refund transaction with a timelock, signs it and sends it to the
hub. The refund transaction is show in Table~\ref{refund-tx}.

\begin{table}
  \centering
  \begin{tabular}{ ll }
    \multicolumn{2}{c}{\bfseries Refund Transaction. Locktime 6 months} \\
    \midrule
    \bfseries Input & \bfseries Output \\
    \midrule
    Fund Tx & \verb|P2WPKH Hub's address| \\
    (Signed by miner) & (R coins) \\
    \midrule
  \end{tabular}
  \caption{Refund transaction signed by miner and held by
    hub.}\label{refund-tx}
\end{table}

With the refund transaction, if the Miner stops responding, the Hub
can get a refund in six months time. However, the hub can be attacked
by sending requests to open new channels and locking up the hub's
liquidity. Hub responds to a miner's channel open request only after
the miner has contributed enough shares. The threshold number of
shares required before opening the channel is a configuration
parameter for the hub. The miner will still receive the payouts for
the shares generated, it is only that the channel opening is delayed.

On receiving the refund transaction, the hub broadcasts the funding
transaction. Once the funding transaction is confirmed, the hub can
start sending payouts to the miner. These payouts are determined in
proportion to the shares found by the miner and included in the DAG as
described in Section~\ref{sec:rewards}.

The payment transactions are updates to the channel where each update
increases the earnings of the miner. The payment transactions are
signed by the hub using the non-cooperating key, \verb|H'|. The
Table~\ref{payment-tx} shows the structure of the payment
transactions.

\begin{center}
  \begin{tabular}{ ll }
    \multicolumn{2}{c}{\bfseries Payment Transaction} \\
    \midrule
    \bfseries Input & \bfseries Output \\
    \midrule
    Fund Tx & \verb|2 H  M  2 CHECKMULTISIG| \\
    (Signed by the hub using H') & \verb|OR| \\
                    & \verb|2 H' M' 2 CHECKMULTISIG + Hash(X)| \\
                    & (Hub: $R - earnings$; Miner: $earnings$) \\
    \midrule
  \end{tabular}
\end{center}

The hub updates the payment channel for each miner with payouts as
determined by the incentives compatible rewards algorithm shown in
Section~\ref{sec:rewards}. In the next section we present how the
payouts are distributed, we then show how the hub and the miners are
disincentivized to cheat.

\subsection{Distributed Payout Algorithm}

Once a miner mines a share that also meets the currently bitcoin
difficulty, the miner immediately broadcasts the block to the bitcoin
network. The coinbase of this block as shown in
Table~\ref{table:coinbase} can now be spent by

\begin{description}
\item[Co-operative Branch:] The miner and the hub by both signing the
  first branch co-operatively.
\item[Hub Branch:] The hub alone, by publishing the pre-image \verb|X|.
\item[Miner Branch:] the miner alone, after waiting for six months.
\end{description}

The proposal by Belcher specifies a payout algorithm that requires the
hub updates the state of all channels, i.e.\ make payment to all
miners. Once it has done so, the miner signs the first branch of the
coinbase transaction and hands the coinbase to the hub. The hub can now
redeem the entire payout.

However, there is a small issue here about how the miner signing the
coinbase can know if the hub has updated the states of all payment
channels. One obvious solution might seem like the miner collects
acknowledgements from all other miners that they have received the
channel update. We still have to deal with the situation where the ACK
from some miners never arrives? Or worse still, if a miner purposely
doesn't send an ACK to stall the pool.

We propose that instead of requiring ACKs from all miners, we extend
the P2P protocol such that all miners store channel updates for all
miners and send them out again if required. This approach is similar
to the use of \emph{inv} and \emph{getdata} messages in bitcoin. We
elaborate further on this in Section~\ref{sec:p2p-protocol}, but for
the purposes of discussion we assume there is a mechanism that allows
a miner to be sure that the hub has updated the states of all payment
channels.

% All miners also send heartbeat messages to hide hub as the source of
% traffic when a block is found. These are all end to end encrypted.

% These messages aren't part of the critical communication channel that
% broadcasts \textsc{work} and \textsc{share}, and is a much lower
% frequency communication.

% What if hub and successful miner collude to not pay anyone?

\subsection{Defecting Does Not Pay}\label{ref:defecting}

Using the above construction of the payment channel and the
distributed payout algorithm, we now show that defecting by the hub or
the miner doesn't pay.

\subsubsection{Hub Defects}\label{ref:hub-defects}

If the hub defects and pays itself from the coinbase it uses the
\verb|hash(X) + Hub P2WPKH| branch of the coinbase. In doing so, the
hub has to reveal the pre-image \verb|X|. With the pre-image
available, all miners can use the
\verb|2 H' M' CHECKMULTISIG + Hash(X)| branch of their payment channel
transactions and close their channels.

The hub could defect on the very first block mined and the miners
would lose all their earnings. But that will end the pool before it
could be useful.

The hub could defect after a few blocks have been mined. In such a
case the miners will receive their fair share of earnings for all
previous blocks, but it would again be the end of life for the pool.

Remember the hub charges fees to fund the payment channels between the
itself and the miners. When the pool ends, the hub loses a profit
making opportunity. We argue that the incentive to defect reduces as
the size of the pool grows.

\subsubsection{Miner Defects}\label{ref:miner-defects}

A miner that found the block could chose to not sign the co-operative
branch of the coinbase. In such a case, the hub will wait for a
timeout period much shorter than the locktime on the miner
branch. This will close all channels and require that all these
channels be opened again.

Such an attack by the miner will hurt miners on the network who have
not yet earned enough payouts to amortise the cost of closing the
channel. However, the miner also loses any payouts since the last
block was found by the pool. A malicious miner could provide a large
portion of the pool's hash rate and then refuse to sign the
co-operative branches, this way it will make the pool defunct, but it
will be an expensive attack. Probably only one the state can afford,
and the only response to such an attack would be for the miners and
the hub to go underground.

\subsubsection{Hub and Miner Collude}\label{ref:collusion}

The hub and the miner could collude where they co-operate to spend the
coinbase to themselves without requiring that the hub pays all other
miners as per the reward schedule. The motivation for the miner is
clear that it earns a big payout, however, such an action by the hub
will end the pool and a stream of future profits for the hub.

%% \subsubsection{Miner Sybil Attack}\label{ref:sybil-attack}

\subsubsection{DDoS on the Hub}\label{ref:ddos-attack}

The hub could be attacked using a distributed denial of service attack
making it unable to process requests to open new channels and to
distributed payouts to miners. Belcher in his proposal suggested the
use of multiple hubs as a defence against such attack. The proposal
also points out that multiple hubs will reduce the liquidity required
to open channels with miners.

According to the Belcher proposal, with multiple hubs available on the
p2p network, miners will open channels to all hubs and receive payouts
from all the hubs. The coinbase is split between hubs in a way that if
any hub defects, the other hubs can still spend the coinbase and split
the block reward between them --- paying the miners appropriately.

Creating a larger coinbase for multiple hubs scales if we can use
Taproot~\cite{bip340,bip341, bip342} once it is activated. The
solution uses staggered timeouts for hubs to spend the coinbase in
case of defecting or hubs under DDoS attacks. We believe this option
is worth exploring once Taproot is enabled and the pool will benefit
from multiple hubs. This will be even more useful as with the multiple
hubs construction, 

We propose a different solution that requires a single hub. The
advantage is a simpler channel construction and the ability to release
the pool without waiting for Taproot activation. The single hub
handling payouts is made indistinguishable from other miners by using
well known responder anonymity~\cite{responder-anonymity:file-sharing,
  liu2010rumor, responder-anonymity:p2p, gap-gnunet} protocols for use
in p2p networks.

\subsection{P2P Network Protocol}\label{sec:p2p-protocol}

The p2p protocol has to enable two types of communications where:

\begin{description}
\item[Shares broadcasts:] Miners broadcast their shares to all other
  miners in the network.
\item[Channel management:] Channel creation and channel state update
  messages exchanged between the miners and the hub.
\end{description}

Even though the two message types seem to be broadcast and unicast
communication, we propose using broadcast for both the communications.

In the first case, an epidemic or gossip based
broadcast~\cite{epidemic-algorithms} algorithm like the one use in
bitcoin and p2pool will meet our goals of making sure all miners
receive all the updates to the DAG of \textsc{SHARE}s. There are a
number of implementation of a gossip based protocols and we don't get
into the details here.

The second case where the miners and the hub have to exchange messages
seems to call for a one to one communication. However, we need to
enable this communication without anyone able to find out which node
is the hub. The hub has to be indistinguishable from other
participants to prevent a DDoS attack on the hub. Our solution is to
use gossip based broadcast of all communication between the hub and
the miner using techniques to hide the source and destination of a
message in a p2p network~\cite{responder-anonymity:file-sharing,
  liu2010rumor, responder-anonymity:p2p, gap-gnunet,
  rumor-riding, p5}. Our solution makes a trade off between latency of
Channel management messages and the hub's anonymity.

We use two different p2p networks for Shares broadcasts and for
Channel management messages. The shares broadcasts network prioritises
lower latency over anonymity, meanwhile the Channel management network
sacrifices latency for hub's anonymity. Since the Channel management
messages are far less frequent than Shares broadcast messages we
think separating the networks and making the latency-anonymity
trade-off is justified.

In this paper, we provide an overview of the approach we will take for
the Channel management p2p network and the details will follow in a
separate document. We next provide the high level approach we take for
the channel management p2p network:

\begin{description}
\item[Epidemic Broadcast] --- All nodes forward all received with
  probability $P_{fwd}$, a system parameter.
\item[Noise] --- All nodes periodically initiate a broadcast, the way
  a hub will do, so that no external observer can use traffic analysis
  to identify the hub.
\item[Message Encryption] --- all messages are encrypted by the sender
  using the hub's public key. Hub encrypts the messages in response
  using the miner's public key included in the received message and
  the miner's \textsc{share}s. The hub's public key is discovered out
  of band.
\item[Pad Messages] --- All encrypted messages are padded to be the
  same size.
\item[Anti-Entropy] --- Occasionally all nodes send a message to
  nodes they are connected to compare the list of channel update
  messages they have seen. If any messages are missing the initiating
  node uses \emph{pull} based mechanism to request missing messages.
\item[Use I2P] --- The channel management p2p network is built on top
  of the I2P~\cite{i2p-censorship-resistance} sublayer to further
  prevent any association between the miner's IP addresses and the
  source of Channel management messages. Recall that the
  \textsc{share}s are broadcast on a separate faster p2p network using
  a different epidemic broadcast protocol.
\end{description}

The above high level approach will generate a lot of traffic on the
channel management network and also results in high latency of message
delivery. The seminal analysis presented in~\cite{epidemic-algorithms}
points to around five messages forwarded by each node if all nodes
forward a received message with 20\% probability. Say we want all
nodes to initiate one channel update message (dummy or genuine) every
block interval, then we get 5000 message transmissions by each node
every ten minutes given a 1000 node network.

There are proposed optimisations to the above described approach in
literature~\cite{responder-anonymity:file-sharing, liu2010rumor,
  responder-anonymity:p2p, gap-gnunet, rumor-riding, p5}, however the
focus of these optimisations is to enable anonymous transfer of large
files. To reduce the bandwidth requirements they reduce the level of
anonymity provided to the receiver of the requests. However, in our
case we are not transferring large files, and we don't want to
compromise of the level of anonymity afforded to the hub.

As a final note, in our initial implementation we will keep the hub
publicly available on a well known IP address, so that all miners can
directly communicate with the hub. As the pool grows, we plan to
switch to the above described p2p network for channel management and
hiding the hub in plain sight.

\section{Future Work}

The proposal presents an approach to enable decentralised mining for
bitcoin. Apart from the work of describing the various components in
detail, we also want to provide results from simulations, formalised
proofs of rewards schemes and possible extensions to using multiple
hubs.

\subsection{Simulations}

Before we work on implementing they system, our next step is to
simulate p2p mining network using ns-3 [ref] and make informed
decisions about how large a network a single hub can support. The
observations we want to make are how large a p2p network can be
sustained without an increase in work lost by miners. Each hub and p2p
network can grow as long as miners are communicate WORK and SHARES
with each other with bounded latency and can limit their lost
work. With a simulation we want to find out the bounds of these.

\subsection{Specifications}

We want to specify the p2p protocols and the message formats for both
the \textsc{share}s propagation and Channel management networks. By
publishing the specifications separate from the source code, we aim to
receive more feedback from the community.

\subsection{Proofs}

We want to use the model presented in~\cite{incentives-compatible} to
provide proofs for how the rewards distribution is incentives
compatible.

\subsection{Multiple Hubs}

We would like to build further on the multiple hubs construction
described by Belcher once Taproot is activated on bitcoin.

\bibliography{proposal} 
\bibliographystyle{acm}

\end{document}
