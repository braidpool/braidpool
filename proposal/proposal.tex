\documentclass{article}

\usepackage[inline]{enumitem}

\usepackage{url}
\usepackage{graphicx}
\graphicspath{ {./figures/} }
\usepackage{amsfonts}% to get the \mathbb alphabet
\usepackage{booktabs}
\usepackage{multirow}


\title{Decentralised Mining Pool for Bitcoin (Draft 0.3)}
\author{}
\date{}

\begin{document}

\maketitle

\begin{abstract}
  Bitcoin P2Pool's usage has steadily declined over the years,
  negatively impacting bitcoin's decentralisation. The variance in
  earnings for miners increases with total hashrate participating in
  the pool, and payouts require a linearly increasing block space with
  the number of miners participating in the pool. We present a
  solution that uses a DAG of shares replicated at all miners. The DAG
  is then used to compute rewards for miners. Rewards are paid out
  using one-way payment channels by an anonymous hub communicating
  with the miners using Tor's hidden services. Using the payment
  channels construction, neither the hub nor the miners can cheat.
\end{abstract}
   
\section{Motivation}

P2Pool~\cite{p2pool:wiki} helps bitcoin's decentralisation by allowing
miners to select the transactions they mine. This avoids any potential
transaction censorship by pool operators. However, the construction
used by P2Pool faces a number of problems that has resulted in miners
abandoning the pool. The most often cited problems are:

\begin{enumerate}
\item large variance in earnings for miners,
\item large number of stale blocks, and
\item large block space requirement for payouts.
\end{enumerate}

The first two problems are a direct consequence of the shares block
rate limited to 30 seconds on the bitcoin p2pool. Intuitively, as the
hash rate participation in the pool goes up, the difficulty for 30
seconds block rate goes up and smaller miners find it harder to find
shares within a 30 second period. This results in an increase in the
variance for shares found by miners and thus for the rewards earned by
the miners. Ethereum's inclusive protocols~\cite{inclusive-protocols}
address the problem by building a DAG of blocks and rewarding blocks
that otherwise would have been left out as stale blocks. As miners get
rewarded for more of their work, the variance on their earnings
reduces enabling smaller miners to remain economically viable.

Payouts in P2Pool are included in the coinbase of the block being
mined. As the number of participants in P2Pool increase the size of
the coinbase transaction increases taking up valuable block space that
the miners could have earned fees from instead. This imposes an upper
bound on the number of participants in P2Pool.

Knowing the challenges faced by P2Pool, we list the goals of a new
decentralised mining pool as:

\begin{enumerate}
\item Reduce variance for miners with increasing pool hash rate.
\item Make payouts to miners with constant size block space
  requirement.
\item Allow miners to select transactions they want to mine, with no
  potential for a censor to deny payouts.
\end{enumerate}

\section{Current Proposals}

TeraHash Coin~\cite{mcelrath:variance}, Jute~\cite{jute} and
Spectre~\cite{spectre} use a DAG for faster block times and focus on
changing the consensus layer of bitcoin. These proposals allow miners
to produce shares that include conflicting transactions and then apply
rules to find a set of transactions acceptable at various cuts of the
DAG, such that the bitcoin consensus rules are not violated.

Braiding the blockchain proposal~\cite{mcelrath:variance} shows how,
smaller more frequent blocks can form a DAG of blocks, called a braid,
with each block pointing to one or more one previous blocks. Blocks in
the braid are called beads and transactions can be repeated in
different beads. The proposal describes how duplicate and double spend
transactions can be resolved to reach a decision on the state of the
ledger at any cut of the DAG.\@

% In the proposal, miners earn a coin native to the braid blockchain,
% called TeraHash Coin. This coin can then be swapped for BTC. The
% proposal doesn't yet define how TeraHash Coin will be swapped for
% BTC. Some of the suggestions under discussion include using atomic
% swaps, burning the TeraHash Coin, or using financial instruments like
% futures of the bitcoin's hash rate to swap TeraHash Coins for BTC.\
% The problems with the atomic swap approach is the need for $O(n)$
% block space in number of miners, as we need two transactions to
% execute a swap with each miner payout. Burning TeraHash Coins for BTC
% requires a trusted third party to execute the burn. Finally, trading
% TeraHash Coins for BTC futures has the problem of introducing
% centralisation on the futures market operator. There is some
% discussion around using DEXes for executing futures contracts, but
% these are initial ideas and we will evaluate them in the future for
% incorporating them into our proposal once they have been developed
% further.

Belcher~\cite{channels-for-rewards} proposes using payment channels to
avoid using block space for making payouts to miners. The construction
uses payment channels between federated hubs to pay miners after a
block has been successfully mined. The payouts are made after a long
period, similar to the 100 blocks requirements for spending from
coinbase transactions. The hub locks in BTC for each miner by opening
payment channels for each of the miners. The construction shows how
both miners and the hub can not cheat and how the funders of the hub
earns rewards for funding the payment channels.

The two ideas of using a DAG and payment channels for rewards
payouts together present a potential path for rebooting P2Pool.

\section{Decentralised Bitcoin Mining}

In this section we describe how a DAG of miner's shares is maintained
by each miner and how the same is used to determine the payout
distribution between miners.

\begin{figure}
  \begin{center}
    \includegraphics[width=1\textwidth]{work-share}
    \caption{Each \textsc{work} generated and shared by a miner is then
      followed by the \textsc{share}s the miner finds.}\label{fig:work-share}
    \end{center}
\end{figure}

\subsection{Work and Share}

Each miner can build their own block, selecting transactions according
to their own criteria. We call this block the \textsc{work} and it is
derived from the results of the \verb|getblocktemplate| bitcoin API
call.

The \textsc{work} definition leaves out the merkle root and the
timestamp. Instead it includes the coinbase transaction so that other
pool participants can verify that the \textsc{work} definition
includes the correct payout scripts. Table~\ref{table:work} shows the
\textsc{work} built by miners. To reduce bandwidth consumption the
transactions field is compressed using compact block
specifications~\cite{compact-blocks}.

\begin{table}
  \centering
  \begin{tabular}{ lp{0.4\linewidth}r }
    \multicolumn{3}{c}{\textsc{work}} \\
    \hline
    Field & Description & Size in bytes \\
    \hline
    Version & Bitcoin block's version field & 4\\
    Previous block & Hash of the previous bitcoin block & 32 \\
    Difficulty & Current bitcoin difficulty & 4 \\
    Coinbase & Coinbase for payment channel setup, see Section~\ref{ref:channels} & 38 \\
    Transactions & Transaction ids included in the block & variable \\
    \textsc{Work} hash & Hash of the above fields & 32 \\
    \hline
  \end{tabular}
  \caption{The structure of \textsc{work} broadcast by
    miners. Timestamp and nonce are left out and are included in
    \textsc{share}s instead.}\label{table:work}
\end{table}

Miners then start mining on \textsc{work} and generate their
\textsc{share}s. Figure~\ref{fig:work-share} shows the relationship
between \textsc{work} and its \textsc{share}s. Each \textsc{work}
created by a miner can result in multiple \textsc{share}s and both the
\textsc{work} and \textsc{share}s are broadcast to the p2p
network. When a miner receives a \textsc{work} from other miners it
validates this \textsc{work} using a local bitcoin node. When a
\textsc{share} is received by a miner it is validated against the
\textsc{work} referenced by the
\textsc{share}. Table~\ref{table:share} shows the structure of
\textsc{share}s broadcast by the miner.


\begin{table}
  \centering
  \begin{tabular}{ lp{0.4\linewidth}r }
    \multicolumn{3}{c}{\textsc{Share}} \\
    \hline
    Field & Description & Size in bytes \\
    \hline
    \textsc{Work} hash & Hash used to identify the \textsc{work} for this share & 32 \\
    Nonce and extra nonce & Nonces used & $4+8$ \\
    Merkle root & New merkle root to include extra nonce & 32 \\
    Timestamp & UNIX timestamp used & 4 \\
    Public keys & Two pubkeys used by the hub for channel management & 66 \\
    $list<$ \textsc{share} hashes $>$ & List of share hashes referenced by this \textsc{share} & $N \times 32$ \\
    Tor Service & PubKey for onion service address, see Section~\ref{sec:hub-miner-communication}. & 32 \\
    \hline
    Total size & \multicolumn{2}{r}{$134 + N \times 32$} \\
    \hline
  \end{tabular}
  \caption{The structure of \textsc{share}s broadcast by miners. Size
    of the \textsc{share} is dependent on the number of shares
    refererenced by a share ($N$).}\label{table:share}
\end{table}

Each \textsc{share} includes the hash of the \textsc{work} the miner
is working on, along with the public keys that the hub uses to setup
payment channels, and finally the pubic key for the Tor hidden service
run by the miner. The Tor hidden service is used for facilitating
anonymous communication between the hub and the miners as described
later in Section~\ref{sec:hub-miner-communication}.

\subsection{DAG of shares}

Miners broadcast their \textsc{share}s to the network using a gossip
protocol and all the miners maintain the DAG of \textsc{share}s as a
replicated database. Each node on the DAG is a \textsc{share}
generated by a miner and all miners track the most recently received
\textsc{share} from all other miners. Miners include hash pointers to
the root nodes of the DAG, ensuring that the DAG is eventually
consistent on all miner nodes. The DAG provides a ``found before''
relationship between shares which enables faster generation of shares
without requiring a single chain of shares.

\begin{figure}
  \begin{center}
    \includegraphics[width=0.75\textwidth]{dag}
    \caption{\textsc{share}s mined between two bitcoin blocks}\label{fig:dag}
  \end{center}
\end{figure}

Figure~\ref{fig:dag} shows a DAG with three miners participating
in the pool, $a$, $b$ and $c$. The nodes in the DAG are the shares
generated by the three miners. Nodes $l$ and $r$ are two valid bitcoin
blocks that have been mined such that they meet bitcoin's difficulty
at the time.

\subsection{Miner Defined Difficulty}\label{sec:share-difficulty}

The nodes in the DAG are \textsc{share}s mined at difficulty level
selected by the miner. This difficulty can be dynamically changed by
the miner after each \textsc{share}, depending on the miner's
observation of the p2p network's hashrate. This dynamic difficulty
adjustment allows miners to adjust the rate at which they produce
\textsc{share}s. This is important as smaller miners can produce low
difficulty shares to match the share rate of the larger miners in the
pool. Even if smaller miners earn a smaller reward, they will earn
these rewards for all their work.

The miner defined difficulty for shares is set in the input of the
coinbase transaction where miners currently also set the
\verb|extraNonce|. This ensures that other miners on the network can
verify the share was mined with the claimed difficulty.

\section{Reward Calculation}\label{sec:rewards}

All participating nodes, including the miners and the hub, maintain a
local replica of the the \textsc{share}s DAG.\@ Each \textsc{share}
includes a reference to the shares the miner was aware of when the
\textsc{share} was found. Each share also encodes the difficulty it
was mined at as part of the nonce as described in
Section~\ref{sec:share-difficulty}. With these in mind, we describe
the rules used to determine the rewards distribution between the
miners.

We use a modified version of PPLNS (Pay Per Last N Shares) to
calculate the reward distribution between miners. Variations of PPLNS
have been shown to be most resistent to pool
hopping~\cite{rosenfeld2011analysis}.

\begin{enumerate}
\item Traverse the DAG in reverse order from the \textsc{share} that
  found a bitcoin block and track the aggregate difficulty included in
  the \textsc{share}s.
\item Once the aggregate difficulty reaches a system defined value,
  stop the DAG traversal.
\item Distribute the reward between miners weighted by the sum of the
  difficulty of all \textsc{share}s found by each miner.
\end{enumerate}

As an example, consider the p2p network of miners $a$, $b$ and $c$
with the DAG of shares as shown in
Figure~\ref{fig:reward-calculation}. In the DAG, two new blocks have
been found, $a_5$ and $b_5$. For the round ending with $a_5$, the last
``N'' shares are $\{a_1..a_5, b_1..b_3\}$. The shares $c_1..c_3$ are
not included in this first round as they are not reachable from
$a_5$. However they are included in the shares for the next round
ending in $b_5$, because their inclusion was needed to reach the total
difficulty for the round. The round ending in $b_5$ therefore rewards
the shares $\{c_1..c_4, b_4, b_5, a_6\}$.

\begin{figure}
  \begin{center}
    \includegraphics[width=1.0\textwidth]{reward-calculation}
    \caption{Reward calculation for two periods. Shares are mined by
      three mines --- $a$, $b$ and $c$. The shares in grey meet the
      bitcoin difficulty at the time they were
      mined.}\label{fig:reward-calculation}
  \end{center}
\end{figure}

% If a miner $a$ doesn't include the \textsc{share}s of miner $b$, it
% signals a failure of communication between the two miners and we
% require that miner $a$ in such a situation should stop including
% references to $b$'s \textsc{share}s. With this rule in place, miners
% are no longer incentivized to find optimal strategies for broadcasting
% their shares to the rest of the miners. They have a simple strategy to
% follow, which is to make sure their shares reach all other miners as
% soon as possible.

% \begin{figure}
%   \begin{center}
%     \includegraphics[width=0.65\textwidth]{isolated-miners}
%     \caption{$a$ discovers a share and the reward is not shared with any
%       other miner.}\label{fig:isolated-miners}
%   \end{center}    
% \end{figure}

% With the above rule in place all miners should include the
% \textsc{share}s discovered by other miners, as otherwise they will be
% excluded by other miners and they will lose the opportunity to be
% rewarded for their work. We identify a degenerative case as ``isolated
% miners'' and argue that miners have no incentives to act in this
% manner. Figure~\ref{fig:isolated-miners} shows a DAG where all three
% miners $a$, $b$ and $c$ are working independently. In such a situation
% when the miner $a$ discovers a share $a_3$ that is a valid bitcoin
% block the reward is not shared with any other miner as $a_3$ does not
% include any references to shares from other miners.

% \subsubsection{Incentive Compatibility}\label{sec:incentive-compatability}

% A reward function is defined as incentive compatible if every miner's
% best response strategy reports full solutions
% immediately~\cite{incentives-compatible}. Where a ``full solution'' is
% a share that meets the bitcoin network's difficulty requirement.

% Given the rules in Section~\ref{sec:rewards}, if a miner finds a
% bitcoin block the miner wants to get maximum reward possible based on
% all the shares it has found and therefore is incentivized to announce
% their \textsc{share} as soon as it finds the block. The longer a miner
% waits to announce the bitcoin block to the bitcoin network, the higher
% the probability that some other miner on the pool will find a
% different bitcoin block that doesn't reward their latest shares that
% haven't reached the other miner. Further still, the rewards
% calculation can be adjusted to give an extra reward to the miner that
% finds the block. This is similar to what p2pool does and Belcher also
% mentions in his proposal.

% \subsubsection{Proportional Payments}\label{sec:proportional-payments}

% Proportional payments~\cite{incentives-compatible} property requires
% that the pool pays miners in proportion to the amount of work they
% have performed. According to our rewards calculation rules, payouts
% are calculated at the end of an epoch and each miner is incentivized
% to include hash pointers to the most recent shares seen from all
% miners. It directly follows that all miners are guaranteed payments
% for their shares that have reached the miner who found the
% block. Therefore the pool satisfies this requirement.

% \subsubsection{Budget Balanced}\label{sec:budget-balanced}

% Budget balanced~\cite{incentives-compatible} property requires that
% the pool operator should never incur a deficit. Again from the rules
% above, since rewards are paid at the end of the epoch, a hub pays out
% rewards without losing or retaining any amount. The hub therefore does
% not make an unaccounted for profit or loss.

% \subsection{Lost Work or Stale Shares}\label{ref:stales}

% \begin{figure}
%   \begin{center}
%     \includegraphics[width=1\textwidth]{shares-propagation.pdf}
%     \caption{Share $c_3$ is a stale share that $c$ earns no reward
%       for.}\label{fig:shares-propagation}
%   \end{center}
% \end{figure}

% We define a share to be stale share if it is not included in two
% consecutive epochs. In Figure~\ref{fig:shares-propagation} the share
% $c_3$ is a stale share as it is not included in the $epoch$s
% terminating in either $a_5$ or $b_5$. We don't reward miners for these
% stale shares as that allows miners to indulge in selfish
% mining~\cite{majority-is-not-enough}. In this section we present rough
% estimates on the work lost by the miners for not being reward for
% these stale shares.

% The details of the p2p network to transmit shares is not yet
% specified, and we don't have simulations results to observe the impact
% of the the pool size both in terms of number of miners and the total
% hash rate of the pool. For now, we present a high level analysis of
% the work lost from stale shares.

% Say $t_{s}$ is the time in seconds to transmit a share to all other
% participants in the network.  Let $t_{b}$ be the time it takes for the
% pool to find a block. It follows that, in the worst case, each miner
% will waste $t_{s}/t_{b}$ amount of work for the time when its work
% fails to reach the miner who finds the next block.

% We work with estimates from observations about block transmission on
% the bitcoin p2p network~\cite{information-propagation} but keep in
% mind that those observations included the time to verify the block and
% that \textsc{share} size at $320kB$ for a p2p network with
% $N = 10,000$ is much smaller than a bitcoin block.

% Given the caveats we proceed to use the same propagation delays from
% the study to capture upper bounds of the work lost and assume a $10$
% second delay for a \textsc{share} to reach all other nodes on the p2p
% network. We also assume that $10,000$ miners pool reaches $2\%$ of the
% global hashrate. This is similar to $12,000$ miners achieving $2\%$
% hashrate by Slushpool~\cite{slushpool}. With this assumption we can
% say that the pool finds a block every $50$ blocks, or approximately
% every $500$ minutes.

% From the above, we get $t_{s} = 10$s and $t_{b} = 500$ minutes and get
% the lost work of around $0.03\%$. Adding the $0.1\%$ fees the miners
% pay to the hub, the miners end up paying $0.13\%$ in costs as compared
% to paying $2-4\%$ to the mining pools, which is still a saving of around
% $93\%$ in fees for the miner when using a decentralised pool.

% From early studies on information dissemination using gossip protocol
% it is known that with $5$ retransmissions of a payload, we can reach
% $99.88\%$ of nodes for a 1000 node
% network\cite{epidemic-algorithms}. Making some pessimisitc assumptions
% here to find out the upper bounds on the work lost. If all the $5$
% transmissions are to far away miners, with a latency of $400ms$, then
% a share will be distributed to $99.9999\%$ of nodes within a $2s$
% interval.

\section{Payment Channels}\label{ref:channels}

We propose using Payment Channels for paying miners based on the
proposal by Belcher~\cite{channels-for-rewards}. The construction of
the payments channels we present is similar to Belcher's construction,
but there are a few changes and we highlight them before describing
the details.

\begin{itemize}
\item In the Belcher proposal, the pool had to predict how much work
  miners will be able to contribute for the next block. Instead our
  DAG based scheme results in an incentives compatible distribution of
  rewards among miners.
\item Belcher proposes using multiple hubs to prevent DDoS attacks on
  hubs, we instead use a single hub, and prevent DDoS attacks by
  keeping the hub hidden using Tor's v3 hidden
  services~\cite{tor-design} for communication between the hub and
  miners.
\end{itemize}

The rest of the construction is similar to the early versions
presented by Belcher --- there is a one-way channel between the hub
and each of the miners. The hub updates the state of each channel with
appropriate reward after each block is found.

% TODO
% Question: How do we know the Hub is not creating coinbases with
% different preimages with each miner?

% Answer: Miners receive the pubkey for all other miners with the
% \textsc{work} announcement. Using this and the structure of the
% coinbase transactions, miners validate that the coinbase matches the
% hub's preimage, and the miner's pubkey.

\subsection{Coinbase}

We use Belcher~\cite{channels-for-rewards} construction where each
miner builds a coinbase transaction that can be spent in one of the
following three ways:

\begin{enumerate}
\item Co-operatively by Hub and Miner, or
\item By the Hub with a hash lock for pre-image \verb|X|, or
\item By the Miner that found the bitcoin block, but after waiting for
  six months.
\end{enumerate}

\begin{table}
  \centering
  \begin{tabular}{ lr }
    \bfseries Coinbase \\
    \midrule
    \verb|2 H M 2 CHECKMULTISIG| & $(cb_1)$ \\
    \verb|OR| \\
    \verb|hash(X) + Hub P2WPKH| & $(cb_2)$ \\
    \verb|OR| \\
    \verb|M and CHECKSEQUENCEVERIFY 6 months| & $(cb_3)$\\ 
    \midrule
  \end{tabular}
  \caption{Coinbase transaction with hub and miner public keys.}\label{table:coinbase}
\end{table}

The scriptPubKey for the above conditions in the coinbase is shown in
Table~\ref{table:coinbase}. These conditions mean that the hub can not
spend the coinbase without revealing the pre-image $X$. This pre-image
is included in the construction of payment channels, as we will see in
the next section. This use of pre-image in both the coinbase and the
payment channel definition guarantees that miners get paid for their
accumulated payouts if the Hub defects and spends through the $cb_2$
branch. We discuss how the miner and the hub don't gain by defecting
in Section~\ref{ref:defecting}.

\subsection{One-way Channels}

One-Way payment channels between hub and all miners allow miners to
receive payouts while consuming a constant size block space in the
bitcoin block they mine. The use of payment channels is what makes the
pool scale without losing block space. The miners therefore avoid
losing the fees that can be earned from the block space.

% Just like in the early versions of proposal by Belcher we use one-way
% payment channels. One-Way payment channels solve the problem of
% aggregating a miner's payouts and require only a single blockchain
% transaction for spending multiple payouts earned from their PoW
% shares.

We use one-way instead of bidirectional payment channels as an initial
implementation. If required we can switch to bidirectional
channels~\cite{poon2016bitcoin} allowing miners to spend their mining
payouts over the lightning network. However, for now, we deliberately
stay away from the complexity of making the hub a lightning node. If
there is interest from miners to use the lightning network, we can
build it in the future.

Each miner has a one-way payment channel with the hub using a two of
two multisig with a time lock of six months. For each payout a miner
receives over the payment channel, the hub will charge an agreed upon
fees between the miner and the hub. Belcher's proposal recommends a
$0.1\%$ fees for the hub.

\subsection{Payment Channel Transactions}

The hub creates a funding channel for each miner and locks $R$ bitcoin
in the channel. The miner then creates a refund transaction, spending
the funding transaction and sending the $R$ bitcoin back to the
hub. The refund transaction has a locktime of six months allowing the
miner to accumulate their payout over the six month period. The
protocol can be extended to allow each miner to agree upon a locktime
with the hub. In this case, the trade-off will be between the fees
charged by the hub and the length of the locktime.

The funding transaction includes an input from the hub and an output
that can be spent in one of the two conditions shown in
Table~\ref{fund-tx}. \verb|H| and \verb|M| are the public keys for hub
and miner, they are called the co-operative keys by Belcher. While
\verb|H'| and \verb|M'| are alternative public keys for hub and miner,
and are called the uncooperative keys in the Belcher proposal.


\begin{table}
  \centering
  \begin{tabular}{ llr }
    \multicolumn{2}{c}{\bfseries Fund Transaction} \\
    \midrule
    \bfseries Input & \bfseries Output \\
    \midrule
    Hub's UTXO & \verb|2 H  M  2 CHECKMULTISIG| & $(f_1)$ \\
    (Signed by the hub) & \verb|OR| \\
                    & \verb|2 H' M' 2 CHECKMULTISIG + Hash(X)| & $(f_2)$ \\
    & (R coins) \\
    \midrule
  \end{tabular}
  \caption{Fund transaction for payment channel between hub and miner.}\label{fund-tx}
\end{table}


The hub doesn't broadcast the funding transaction, instead it waits
for the miner to create refund transaction. This is the same as any
other timelocked one-way channel construction, i.e.\ the miner creates
a refund transaction with a timelock, signs it and sends it to the
hub. The refund transaction is shown in Table~\ref{refund-tx}.

\begin{table}
  \centering
  \begin{tabular}{ ll }
    \multicolumn{2}{c}{\bfseries Refund Transaction. Locktime 6 months} \\
    \midrule
    \bfseries Input & \bfseries Output \\
    \midrule
    Fund Tx & \verb|P2WPKH Hub's address| \\
    (Signed by miner) & (R coins) \\
    \midrule
  \end{tabular}
  \caption{Refund transaction signed by miner and held by
    hub.}\label{refund-tx}
\end{table}

With the refund transaction, if the miner stops responding, the hub
can get a refund in six months time. However, the hub can be attacked
by sending requests to open new channels and locking up the hub's
liquidity. We resolve this by requiring the hub to open a channel to a
miner only after it has contributed enough shares. This can be a
parameter of the pool instantiation, with values anywhere between one
to 100 bitcoin blocks. The threshold number of shares required before
opening the channel is a configuration parameter for the hub. The
miner will still receive the payouts for the shares generated, it is
only that the channel opening is delayed.

On receiving the refund transaction, the hub broadcasts the funding
transaction. Once the funding transaction is confirmed, the hub can
start sending payouts to the miner. These payouts are determined in
proportion to the shares found by the miner and included in the DAG as
described in Section~\ref{sec:rewards}.

The payment transactions are updates to the channel where each update
increases the earnings of the miner. The payment transactions are
signed by the hub using the non-cooperating key, \verb|H'|.
Table~\ref{payment-transaction} shows the structure of the payment
transactions. In the next section we present how the payouts are
distributed, we then show how the hub and the miners are
dis-incentivized to cheat.


\begin{table}
  \centering
  \begin{tabular}{ llr }
    \multicolumn{2}{c}{\bfseries Payment Transaction} \\
    \midrule
    \bfseries Input & \bfseries Output \\
    \midrule
    Fund Tx & \verb|2 H  M  2 CHECKMULTISIG| & $(p_1)$ \\
    (Signed by the hub using H') & \verb|OR| \\
                    & \verb|2 H' M' 2 CHECKMULTISIG + Hash(X)| \\
                    & (Hub: $R - earnings$; Miner: $earnings$) & $(p_2)$\\
    \midrule
  \end{tabular}
  \caption{Payment channel update transaction sent from hub to the
    miner.}\label{payment-transaction}
\end{table}

\subsection{Channel Updates for Payouts}\label{sec:channel-update}

Once a miner mines a share that also meets the current bitcoin
difficulty, the miner immediately broadcasts the block to the bitcoin
network. The coinbase of this block is shown in
Table~\ref{table:coinbase} and can now be spent by one of the following
branches:

\begin{description}
\item[Co-operative Branch:] $(cb_1)$ The miner and the hub by both signing the
  first branch co-operatively.
\item[Hub Branch:] $(cb_2)$ The hub alone, by publishing the pre-image \verb|X|.
\item[Miner Branch:] $(cb_3)$ the miner alone, after waiting for six months.
\end{description}

The proposal by Belcher specifies a payout algorithm that requires the
hub updates the state of all channels, i.e.\ make payment to all
miners. Once it has done so, the miner who found the block signs the
first branch of the coinbase transaction and hands the coinbase to the
hub. The hub can now redeem the entire payout. We use the same
approach, requiring the miner to receive all channel updates from the
hub before signing the co-operative branch. The miner first verifies
that that the channel updates are properly signed by the hub, and
there is one update for all miners in proportion to the rewards
distribution algorithm.

\subsection{Defecting Does Not Pay}\label{ref:defecting}

Using the above construction of the payment channel and the
distributed payout algorithm, we now show that defecting by the hub or
the miner doesn't pay.

\subsubsection{Hub Defection}\label{ref:hub-defects}

If the hub defects and pays itself from the coinbase it uses the
$cb_2$ branch of the coinbase. In doing so, the hub has to reveal the
pre-image \verb|X|. With the pre-image available, all miners will use
the $p_2$ branch of their payment channel transactions and close their
channels receiving all the payouts earned up to that block.

It is possible that the hub defects on the very first block mined and
the miners lose their earnings for that single block. But that will
end the pool before it could be useful.

The hub could defect after a few blocks have been mined. In such a
case the miners will receive their fair share of earnings for all
previous blocks, but it will again be the end of life for the pool.

We recall that the hub charges fees to fund the payment channels
between the itself and the miners. When the pool ends, the hub loses a
profit making opportunity. We argue that the incentive to defect
reduces as the size of the pool grows.

It is worth noting that if the hub is co-opted, all it can do is deny
miners a payout for the number of blocks the pool delays the payments
by. We introduced the idea of this ranging from one to 100
blocks. Further, if the hub is co-opted, all miners will immediately
close their channels, collect their payouts and re-organise with a
different hub.

\subsubsection{Miner Defects}\label{ref:miner-defects}

A miner that found the block could chose to not sign the co-operative
branch $(cb_1)$ of the coinbase. In such a case, the hub will wait for
a timeout period much shorter than the locktime on the miner branch
$(cb_3)$ and receive the payout using the $(cb_2)$ branch of the
coinbase. In response to this, all other miners will close their
channels by signing the $(p_2)$ branch of the payment transaction by
using the pre-image \verb|X| included in $(cb_2)$ broadcast by the
hub. This will close all channels and require that all these channels
are opened again. Such an attack by the miner will hurt miners on the
network who have not yet earned enough payouts to amortise the cost of
closing the channel.

Say a miner starts participating in the pool, and after $N$ blocks
successfully mines a bitcoin block, but refuses to sign the
co-operative branch of the coinbase allowing the hub to claim the
coinbase reward. In such a case all miners have to close their
channels and claim the rewards they have earned. Miners who recently
joined the pool stand to lose the most because of the forced closure
of channels.

However, the miner also loses any payouts for all the shares it mined
since the last block was found by the pool. A malicious miner could
contribute a large portion of the pool's hash rate and then refuse to
sign the co-operative branch. This will disrupt the functioning of the
pool and a well funded censor could be willing to execute such an
attack. The only defence here is that the miners re-organise and start
a different instance of the pool.

\subsubsection{Hub and Miner Collude}\label{ref:collusion}

The hub and the miner could collude where they co-operate to spend the
coinbase to themselves without requiring that the hub pays all other
miners as per the reward schedule. The motivation for the miner is a
big payout it can receive. However, such an action by the hub will
end the pool and the stream of future profits for the hub.

%% \subsubsection{Miner Sybil Attack}\label{ref:sybil-attack}

\subsubsection{DDoS on the Hub}\label{ref:ddos-attack}

The hub can be attacked using a distributed denial of service attack
rendering it unable to process requests to open new channels and to
distribute payouts to miners. Belcher in his proposal suggested the
use of multiple hubs as a defence against such an attack. The proposal
also points out that using multiple hubs will also reduce the
liquidity required to open channels with miners.

According to Belcher's proposal, with multiple hubs available on the
p2p network, miners will open channels to all hubs and receive payouts
from all of the hubs. The coinbase is split between hubs in such a way
that if any hub defects, the other hubs can still spend the coinbase
and split the block reward between them. In this way all miners
receive payouts in proportion to the shares contributed by them.

Creating a larger coinbase for multiple hubs scales if we can use
Taproot~\cite{bip340,bip341, bip342} once it is activated. The
solution uses staggered timeouts for hubs to spend the coinbase in
case of hubs defecting or coming under DDoS attacks. The only problem
is that in case of hubs defecting, the staggered timeouts can result
in a large wait for the payouts to be distributed.

We propose a different solution that requires a single hub. The
advantage is a simpler channel construction and allow us to build the
mining pool without waiting for Taproot activation. The single hub
approach required that all miners run a Tor v3 hidden service and
publish the public key used to generate the service's address. The
public key is published in each share the miner publishes. The hub can
now contact the miners anonymously as a client and therefore avoids
DoS attacks on the hub or on the miners hidden services.

\subsection{Anonymous Hub Miner
  Communication}\label{sec:hub-miner-communication}

The hub and the miner need to exchange messages at the time of channel
creation as well updating the channel state when the hub makes a
payout to the miners. The channel management messages are exchanged
infrequently as compared to the \textsc{share}s broadcast messages and
don't have the same timeliness requirements as the shares
broadcasts. We propose adopting a solution that doesn't compromise on
the hub's anonymity and makes a trade-off with increased latency of
the channel management messages. We use Tor's hidden services to avoid
compromising the hub's anonymity.

Miners will run a Tor hidden service next to their mining
controller. Once the service is ready, miners include their service's
public key in their share headers. The hub can then communicate
anonymously with the miners to create payment channels and to update
the payment channels. The hub also publishes a well known public key
on a publicly accessible location like a github repository or an IRC
channel. The hub then participates in the shares broadcast p2p network
and identifies a new miner along with the new miner's hidden service
address. Once the miner has mined for a certain time period, the hub
contacts the miner to set up the payment channels. The hub signs all
their messages using their well known public key.

To send payouts as channel state updates, the hub sends the updates to
all miners using their hidden service address. The hub sends all
miners the updates to all payment channels, allowing the miner who
discovered a block to verify that the hub has paid everyone, before it
sends the channel update transaction to the hub as described in
Section~\ref{sec:channel-update}.

\section{Future Work}

We have not covered the goal of providing tools for building hashrate
futures markets in this paper. The initial ideas for the same are
captured in a
gist\footnote{https://gist.github.com/kulpreet/19927c7188a4224ce2de43efb3c69370}. Further,
Taproot provides other possibilities for enabling trading of shares
published on a p2p network of miners.

% Our proposal presents an approach to enable decentralised mining for
% bitcoin. Apart from the work of describing the various components in
% detail, we also want to provide results from simulations, formalised
% proofs of rewards schemes and possible extensions to using multiple
% hubs.

% Before we work on implementing the system, our next step is to
% simulate p2p mining network and make informed decisions about how
% large a network a single hub can support. The observations we want to
% make are how large a p2p network can be sustained without an increase
% in work lost by miners. Each instance of the pool uses a single hub
% and the p2p network of miners can grow as long as miners can
% communicate \textsc{work} and \textsc{share}s with each other within
% bounded latencies. We want to find the bounds of these limits using a
% simulation.

% We will also specify the p2p protocols and the message formats for
% both the \textsc{share}s propagation and channel management
% networks. By publishing the specifications separate from the source
% code, we aim to receive more feedback from the community. We want to
% use the model presented in~\cite{incentives-compatible} to provide
% proofs for how the rewards distribution is incentives compatible. We
% would like to build further on the multiple hubs construction
% described by Belcher once Taproot is activated on bitcoin.

\section{Conclusion}

We propose a new decentralised mining pool for Bitcoin that enables
miners to build their own blocks and doesn't increase the variance nor
the blockspace required for payouts as the mining pool grows.

% \section{Acknowledgements}


\bibliography{proposal} 
\bibliographystyle{acm}

\end{document}
